% =======================
% BÀI TẬP (pdflatex, không thêm gói)
% =======================

% Dự phòng khi không nạp amsmath/amssymb
\providecommand{\tanh}{\mathrm{tanh}}
\providecommand{\odot}{\mathbin{\circ}}

% ---- Tiêu đề trang ----
\begin{center}
  {\LARGE\bfseries ĐỀ BÀI}\\[6pt]
  {\normalsize Thiết kế mô hình Encoder--Decoder dựa trên LSTM cho bài toán phân tích cảm xúc (Positive, Negative, Neutral)}
\end{center}
\addcontentsline{toc}{section}{Đề bài: Encoder--Decoder LSTM cho phân tích cảm xúc}


% ---- Mục tiêu bài tập ----
\vspace{6pt}
{\large\bfseries Mục tiêu bài tập}\par
\HRule

Sinh viên sẽ:
\begin{enumerate}[label=\arabic*., leftmargin=1.2cm]
  \item \textbf{Thiết kế} một mô hình \textbf{Encoder--Decoder LSTM} để phân loại cảm xúc của câu.
  \item \textbf{Vẽ sơ đồ kiến trúc} mô hình (trên giấy hoặc công cụ vẽ).
  \item \textbf{Mô tả chi tiết tham số} (kích thước embedding, hidden size, trọng số, bias…).
  \item \textbf{Tính toán thủ công} từng bước \textbf{forward pass}:
    \begin{itemize}
      \item Encoder xử lý từng từ $\rightarrow$ tạo \textit{context vector}.
      \item Decoder nhận context vector $\rightarrow$ sinh nhãn cảm xúc.
    \end{itemize}
  \item \textbf{Giải thích cơ chế hoạt động} của mô hình.
  \item \textbf{Liên hệ} với đồ án dịch máy (tương đồng và khác biệt).
\end{enumerate}

% ---- Yêu cầu nộp bài ----
\vspace{6pt}
{\large\bfseries Yêu cầu nộp bài}\par
\HRule
\begin{center}
\renewcommand{\arraystretch}{1.3}
\begin{tabular}{|p{7.1cm}|p{6.1cm}|}
\hline
\textbf{Nội dung} & \textbf{Hình thức} \\ \hline
1. Sơ đồ kiến trúc mô hình & Dùng công cụ vẽ (Draw.io hoặc tương đương) \\ \hline
2. Bảng mô tả tham số & Bảng rõ ràng, mô tả kích thước và ý nghĩa \\ \hline
3. Tính toán thủ công & Trình bày từng bước, có công thức \\ \hline
4. Báo cáo giải thích & Giải thích kiến trúc, cơ chế hoạt động \\ \hline
5. File nộp & 1 file PDF hoàn chỉnh \\ \hline
\end{tabular}
\end{center}

% ---- Dữ liệu giả lập ----
\vspace{10pt}
{\large\bfseries Dữ liệu giả lập (đã cho sẵn)}\par
\HRule

\begin{center}
\renewcommand{\arraystretch}{1.3}
\begin{tabular}{|p{5cm}|p{8cm}|}
\hline
\textbf{Thành phần} & \textbf{Giá trị} \\ \hline
\textbf{Câu đầu vào} & ``I love this movie!'' (4 từ) \\ \hline
\textbf{Embedding} & Mỗi từ $\rightarrow$ \textbf{vector 2 chiều} \\ \hline
\end{tabular}
\end{center}

% Bảng từng từ -> vector
\begin{center}
\renewcommand{\arraystretch}{1.4}
\begin{tabular}{|c|c|}
\hline
\textbf{Từ} & \textbf{Vector} \\ \hline
I &
$\displaystyle x_1=
\left[\!
\begin{array}{c}
0.5\\
0.1
\end{array}
\!\right]$ \\ \hline
love &
$\displaystyle x_2=
\left[\!
\begin{array}{c}
0.8\\
0.7
\end{array}
\!\right]$ \\ \hline
this &
$\displaystyle x_3=
\left[\!
\begin{array}{c}
0.2\\
0.3
\end{array}
\!\right]$ \\ \hline
movie &
$\displaystyle x_4=
\left[\!
\begin{array}{c}
0.6\\
0.8
\end{array}
\!\right]$ \\ \hline
\end{tabular}
\end{center}

% ===== Bảng dữ liệu chi tiết (chuẩn như hình) =====
\begin{center}
\renewcommand{\arraystretch}{1.35} % giãn dòng cho đẹp
\begin{tabular}{|p{5cm}|p{8cm}|}
\hline
\textbf{Thành phần} & \textbf{Giá trị} \\ \hline

~ &
``this'': $x_3 =
\displaystyle
\left[\!
\begin{array}{c}
0.2\\
0.3
\end{array}
\!\right]$ \\ \hline

~ &
``movie!'': $x_4 =
\displaystyle
\left[\!
\begin{array}{c}
0.6\\
0.8
\end{array}
\!\right]$ \\ \hline

\textbf{Nhãn ground truth} &
\textbf{Positive} $\rightarrow$ one-hot: $\displaystyle
y=\left[\!
\begin{array}{c}
1\\
0\\
0
\end{array}
\!\right]$ \\ \hline

\textbf{Trạng thái ban đầu} &
$\displaystyle h_0=c_0=
\left[\!
\begin{array}{c}
0\\
0
\end{array}
\!\right]$ \\ \hline

\textbf{Hidden size} &
\textbf{2} (để dễ tính thủ công) \\ \hline

\textbf{Số lớp đầu ra} &
\textbf{3} (Positive, Negative, Neutral) \\ \hline
\end{tabular}
\end{center}



% ---- Công thức LSTM ----
\vspace{10pt}
{\large\bfseries Công thức LSTM chuẩn (4 cổng)}\par
\HRule
\[
f_t=\sigma(W_{xf}x_t+W_{hf}h_{t-1}+b_f),\qquad
i_t=\sigma(W_{xi}x_t+W_{hi}h_{t-1}+b_i)
\]
\[
\tilde{c}_t=\tanh(W_{xc}x_t+W_{hc}h_{t-1}+b_c),\qquad
o_t=\sigma(W_{xo}x_t+W_{ho}h_{t-1}+b_o)
\]
\[
c_t=f_t\odot c_{t-1}+i_t\odot \tilde{c}_t,\qquad
h_t=o_t\odot \tanh(c_t)
\]

% ---- Nhiệm vụ chi tiết ----
\vspace{8pt}
{\large\bfseries Nhiệm vụ chi tiết}\par
\HRule

\textbf{1. Vẽ sơ đồ kiến trúc mô hình (2 điểm)}\\
\emph{Vẽ đầy đủ:}
\begin{itemize}
  \item Input $\rightarrow$ Word Embedding $\rightarrow$ Encoder LSTM (4 bước) $\rightarrow h_4,c_4$.
  \item $h_4,c_4 \rightarrow$ Decoder LSTM (1 bước) $\rightarrow h'_1$.
  \item $h'_1 \rightarrow W_{hy} \rightarrow$ Softmax $\rightarrow$ Output (3 lớp).
\end{itemize}
Ghi chú tên biến, kích thước, luồng dữ liệu. Dùng mũi tên, hộp, chú rõ ràng.

\vspace{6pt}
\textbf{2. Mô tả các tham số (1 điểm)}\\
Tạo bảng:
\begin{center}
\renewcommand{\arraystretch}{1.3}
\begin{tabular}{|c|c|c|}
\hline
\textbf{Tham số} & \textbf{Kích thước} & \textbf{Mô tả} \\ \hline
$x_t$ & ? & ... \\ \hline
$h_t,c_t$ & ? & ... \\ \hline
$W_{xf},W_{hf},\ldots$ & ? & ... \\ \hline
$W_{hy}$ & ? & ... \\ \hline
\textbf{Hidden size} & ? & ... \\ \hline
\textbf{Số lớp} & ? & ... \\ \hline
\end{tabular}
\end{center}

\vspace{6pt}
\textbf{3. Tính toán thủ công: Forward pass (5 điểm)}\\
\emph{Bước 1: Encoder -- Xử lý từng từ}
\begin{itemize}
  \item Tính từng cổng cho $t=1$ (từ “I”).
  \item Tính tương tự cho $t=2,3,4$.
  \item Ghi tất cả phép tính: nhân ma trận, $\sigma$, $\tanh$, $\odot$.
  \item Tìm $h_4,c_4$ (context vector).
\end{itemize}
\emph{Bước 2: Decoder -- Phân loại}
\begin{itemize}
  \item Dùng $h'_0=h_4,\; c'_0=c_4$.
  \item Tính \textbf{1 bước} LSTM của Decoder $\rightarrow h'_1$.
  \item $z=W_{hy}h'_1$, tính softmax $\rightarrow \hat{y}$ (3 giá trị).
  \item Tính cross-entropy loss: $L=-\sum_i y_i\log\hat{y}_i$.
\end{itemize}

\vspace{6pt}
\textbf{4. Giải thích cơ chế hoạt động (2 điểm)}\\
Trả lời:
\begin{enumerate}[label=\arabic*)]
  \item Encoder làm gì với đầu vào? Vì sao $h_4,c_4$ gọi là \textit{context vector}?
  \item Decoder làm gì? Vì sao chỉ cần 1 bước?
  \item Teacher forcing có dùng không? Vì sao?
  \item Loss function đo lường điều gì?
  \item Liên hệ với đồ án dịch máy: điểm tương đồng và khác biệt (output, mục tiêu).
\end{enumerate}
